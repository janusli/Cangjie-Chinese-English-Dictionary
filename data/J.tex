\chapter{十部}

%%%%%%%%%%%%%%%%%%%%%%%%%%%%
%%%%%%%%%%%%%%%%%%%%%%%%%%%%
\begin{Lemma} %
	\centry[10][人]{倌}[人十口口][]{gu\=an}{}
	\Etymology{形聲。}\pro{}\lipsum
	%%%%%%%%%%%%%%%%%%%
	\begin{mainenum}
		\mitem  \en{}
		\E{\e \e \e }
		%%%%%%%%%%%%%%%%%
		\mitem  \en{}
		\E{\e \e \e }
		%%%%%%%%%%%%%%%%%
		\mitem  \en{}
		\E{\e \e \e }
		%%%%%%%%%%%%%%%%%
	\end{mainenum}
\end{Lemma}
%%%%%%%%%%%%%%%%%%%%%%%%%%%%
%%%%%%%%%%%%%%%%%%%%%%%%%%%%
\clearpage
%%%%%%%%%%%%%%%%%%%%%%%%%%%%
\begin{Lemma} %
	\centrya[6][宀]{字}[JND][3040\SB{7}]{zi\SP{4}}{zi\SP{6}}
	\Etymology{會意。形聲。}古文:𡥜𥤪。\emph{說文}:乳也,从子在宀下,子亦聲。\pro{\emph{廣韻}:疾置切,去志,從。\emph{正韻}:疾二切,𠀤音自。}%\lipsum
	%%%%%%%%%%%%%%%%%%%
	\begin{mainenum}
		\mitem 懷\textsl{孕};生\textsl{育}。\en{conceive},\en{pregnant}
		\E{\e \emph{易}\emph{屯}:女子貞不字,十年乃字。\e \emph{墨子}\emph{節用上}:後聖王之法十年,若純三年而字,子生可以二三年矣。}
		%%%%%%%%%%%%%%%%%
		\mitem  乳\textsl{哺};\textsl{養}育。 \en{bring up}, \en{rear}
		\E{\e \emph{詩}\emph{大雅}\emph{生民}:誕置之隘巷,牛羊腓字之。\e \emph{左傳}\emph{昭公十一年}:其僚無子,使字\emph{敬叔}。\e \textsf{林紓}\emph{陳德齋墓志銘}:君誕時,母\textsf{劉太宜人}乏食,幾不能字之。}
		%%%%%%%%%%%%%%%%%
		\mitem  引申指\textsl{敎}化。\en{educate},\en{teach}
		\E{\e \emph{隋}\emph{盧思道}\emph{遼陽山寺願文}:皇帝體膺上哲,運鍾下武,以至德字黔首。}
		%%%%%%%%%%%%%%%%%
		%%%%%%%%%%%%%%%%%
		\mitem 撫\textsl{愛};愛護。 \en{love},\en{affection}
		\E{\e \emph{書}\emph{康誥}:于父不能字厥子,乃疾厥子。\emph{孔}傳:於為人父不能字愛其子,乃疾惡其子,是不慈。\e \emph{左傳}\emph{成公四年}:楚雖大,非吾族也,其肯字我乎?\emph{杜預}注:字,愛也。}
		%%%%%%%%%%%%%%%%%		
		%%%%%%%%%%%%%%%%%
		\mitem 文字。 \en{character},\en{letter}
		\E{\e \textsf{漢}\textsf{許慎}\emph{說文解字}\emph{敍}:\textsf{倉頡}之初作書,蓋依類象形,故謂之文。其後形聲相益,即謂之字。字者,言孳乳而浸多也。\e \textsf{南朝}\textsf{梁}\textsf{劉勰}\emph{文心雕龍}\emph{物色}:「皎日」「嘒星」,一言窮理;「參差」「沃若」,兩字窮形。}
		%%%%%%%%%%%%%%%%%		
		%%%%%%%%%%%%%%%%%
		\mitem 字\textsl{音}。 \en{pronunciation},\en{accent}
		\E{\e \textsf{宋}\textsf{辛棄疾}\emph{好事近}\emph{送李復州致一席上和韻}:和淚唱\emph{陽關},依舊字嬌聲穩。\e \textsf{元}\textsf{喬吉}\emph{一枝花}\emph{合箏}:遲疾纖巧隨摳掐無些兒病,腔兒穩,字兒正。}
		%%%%%%%%%%%%%%%%%
		%%%%%%%%%%%%%%%%%
		\mitem 字體。 \en{font}
		\E{\e \emph{晉書}\emph{衛恒傳}:或曰,\textsf{邈}所定乃隸字也。\e 顏字厚重,柳字大氣。}
		%%%%%%%%%%%%%%%%%		
		%%%%%%%%%%%%%%%%%
		\mitem 字眼;詞語。 \en{word}
		\E{\e \emph{初刻拍案驚奇}卷二二:若要覓衣食,須把個「官」字兒擱起,照着常人,傭工做活,方可度日。}
		%%%%%%%%%%%%%%%%%		
		%%%%%%%%%%%%%%%%%
		\mitem 引申指運用文字的能力、文化水平。 \en{literacy}
		\E{\e \emph{兒女英雄傳}第十一回:我本來字兒也沒你的深,主意也沒你的巧妙。}
		%%%%%%%%%%%%%%%%%		
		%%%%%%%%%%%%%%%%%
		\mitem 書信、條據等文字材料。 \en{letter},\en{message},\en{information}
		\E{\e \textsf{唐}\textsf{杜甫}\emph{登岳陽樓}:親朋無一字,老病有孤舟。\e \textsf{宋}\textsf{范成大}\emph{戲贈少梁}:秋來合有相思字,會待風前片葉看。\e \emph{儒林外史}第一回:這一回小婿再去,托敝親家寫一封字來,去晉謁晉謁\textsf{危}老先生。\e 立字爲憑。}
		%%%%%%%%%%%%%%%%%		
		%%%%%%%%%%%%%%%%%
		\mitem 寫字。 \en{write}
		\E{\e \textsf{宋}\textsf{蘇軾}\emph{與歐陽晦夫書}:\emph{地獄變相}已跋其後,可詳味之,似有補於世者,並字數紙納去。}
		%%%%%%%%%%%%%%%%%		
		%%%%%%%%%%%%%%%%%
		\mitem 書法作品。 \en{calligraphy}
		\E{\e \textsf{魯迅}\emph{書信集}\emph{致增田涉}:今天已將我寫的字兩件托\textsf{內山}老闆寄上,\textsf{鐵研翁}的一幅,因先寫,反而拙劣。}
		%%%%%%%%%%%%%%%%%		
		%%%%%%%%%%%%%%%%%
		\mitem 人的表字。在本名外所取的與本名意義相關的另一名字。 \en{alias}
		\E{\e \emph{史記}\emph{孔子世家}:\textsf{孔子}生\textsf{鯉},字\textsf{伯魚}。\e \textsf{北齊}\textsf{顏之推}\emph{顏氏家訓}\emph{風操}:古者名以正體,字以表德。}
		%%%%%%%%%%%%%%%%%	
		%%%%%%%%%%%%%%%%%
		\mitem 取表字。\en{alias}
		\E{\e \emph{楚辭}\emph{離騷}:名余曰\textsf{正則}兮,字余曰\textsf{靈均}。\e \emph{禮記}\emph{曲禮上}:男子二十,冠而字……女子許嫁,笄而字。}
		%%%%%%%%%%%%%%%%%			
		%%%%%%%%%%%%%%%%%
		\mitem 取\textsl{名}。 \en{name}
		\E{\e \textsf{唐}\textsf{李紳}\emph{鶯鶯歌}:綠窗嬌女字鶯鶯,金雀婭鬟年十七。\e \e }
		%%%%%%%%%%%%%%%%%		
		%%%%%%%%%%%%%%%%%
		\mitem 用表字稱呼。 \en{}
		\E{\e \emph{周書}\emph{伊婁穆傳}:嘗入白事,太祖望見悅之,字之曰「\textsf{奴干}作儀同面見我矣」。}
		%%%%%%%%%%%%%%%%%			
		%%%%%%%%%%%%%%%%%
		\mitem 稱女子許配,出嫁。 \en{}
		\E{\e \textsf{宋}\textsf{葉適}\emph{林伯和墓志銘}:鄰女將字而孤,養視如己子,擇對嫁之。\e \emph{續資治通鑒}\emph{宋太宗至道二年}:\textsf{高麗}國王\textsf{王治}請婚於\textsf{遼},\textsf{遼}許以\textsf{東京}留守\textsf{蕭恒德}女字之。\e 待字閨中。}
		%%%%%%%%%%%%%%%%%			
		%%%%%%%%%%%%%%%%%		
	\end{mainenum}
\end{Lemma}
%%%%%%%%%%%%%%%%%%%%%%%%%%%%
%%%%%%%%%%%%%%%%%%%%%%%%%%%%
%\begin{Lemma} %
%	\centry[20][宀]{𡫸}[JNGR][]{ni\SP{4}}{ni}
%	\Etymology{}\pro{}\emph{字彙補}\emph{宀部}:寧吉切,音昵。出釋典,切身。\en{}
%%%%%%%%%%%%%%%%%%%
%\end{Lemma}
%%%%%%%%%%%%%%%%%%%%%%%%%%%%
%%%%%%%%%%%%%%%%%%%%%%%%%%%%
\begin{Lemma} %
	\centry[14][古]{嘏}[JRRSE][4764\SB{7}]{gu\SP{3}}{gu\SP{2}}
	\Etymology{}\pro{\emph{廣韻}:古疋切,上馬,見。}\en{big}。\emph{詩}\emph{周頌}:伊嘏\textsf{文王},既右饗之。\textsf{孔穎達}疏引\textsf{王肅}曰:維天乃大\textsf{文王}之德,既佑助而歆饗之。
	%%%%%%%%%%%%%%%%%%%
\end{Lemma}
%%%%%%%%%%%%%%%%%%%%%%%%%%%%
%%%%%%%%%%%%%%%%%%%%%%%%%%%%
\begin{Lemma} %
	\centrya[14][宀]{寥}[JSSH][3020\SB{2}]{liao\SP{2}}{liu\SP{4}}\Etymology{}\pro{\emph{廣韻}:落蕭切,平蕭,來。\emph{廣韻}:郎擊切,入錫,來。}\en{empty}。\emph{老子}:有物混成,先天地生,寂兮寥兮。獨立不改。\textsf{孔穎達}疏引\textsf{王肅}曰:維天乃大\textsf{文王}之德,既佑助而歆饗之。
	%%%%%%%%%%%%%%%%%%%
\end{Lemma}
%%%%%%%%%%%%%%%%%%%%%%%%%%%%








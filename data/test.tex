


%%===================================
%%===================================
\begin{Lemma} %
	\centry[][]{}[][]{\SP{}}{\SP{}}
	\pro{}\lipsum
	%%======================
	\begin{mainenum}
		\mitem  \en{}
		\E{\e \e \e }
		%%======================
		\mitem  \en{}
		\E{\e \e \e }
		%%======================
		\mitem  \en{}
		\E{\e \e \e }
		%%======================
	\end{mainenum}
\end{Lemma}
%%===================================
%%===================================




%%===================================
%%===================================
\begin{Lemma} %
	\centry[][]{}[][]{\SP{}}{\SP{}}
	\pro{}\pro{}
	%%======================
	\begin{mainenum}
		\mitem  \en{}
		\E{\e \e \e }
		%%======================
		\mitem  \en{}
		\E{\e \e \e }
		%%======================
		\mitem  \en{}
		\E{\e \e \e }
		%%======================
		\mitem  \en{}
		\E{\e \e \e }
		%%======================
		\mitem  \en{}
		\E{\e \e \e }
		%%======================
		\mitem  \en{}
		\E{\e \e \e }
		%%======================
	\end{mainenum}
\end{Lemma}
%%===================================
%%===================================


%%%%%%%%%%%%%%%%%%%%%%%%%%%%
%%%%%%%%%%%%%%%%%%%%%%%%%%%%
\begin{Lemma} %
	\centry[總筆畫數][所屬部首]{本字}[倉頡碼][四角碼]{當代國語拼音}{粵語拼音}
	\pro{古書注音一}\pro{古書注音二}
	%%%%%%%%%%%%%%%%%%%%%%%%%%%%
	\begin{mainenum}
		\mitem  \en{}
		\E{\e \e \e }
		%%%%%%%%%%%%%%%%%%%%%%%%%%%%
		\mitem  \en{}
		\E{\e \e \e }
		%%%%%%%%%%%%%%%%%%%%%%%%%%%%
		\mitem  \en{}
		\E{\e \e \e }
		%%%%%%%%%%%%%%%%%%%%%%%%%%%%
		\mitem  \en{}
		\E{\e \e \e }
		%%%%%%%%%%%%%%%%%%%%%%%%%%%%
		\mitem  \en{}
		\E{\e \e \e }
		%%%%%%%%%%%%%%%%%%%%%%%%%%%%
		\mitem  \en{}
		\E{\e \e \e }
		%%%%%%%%%%%%%%%%%%%%%%%%%%%%
		\mitem  \en{}
		\E{\e \e \e }
		%%%%%%%%%%%%%%%%%%%%%%%%%%%%	
		\mitem  \en{}
		\E{\e \e \e }
		%%%%%%%%%%%%%%%%%%%%%%%%%%%%
		\mitem  \en{}
		\E{\e \e \e }
		%%%%%%%%%%%%%%%%%%%%%%%%%%%%	
		\mitem  \en{}
		\E{\e \e \e }
		%%%%%%%%%%%%%%%%%%%%%%%%%%%%	
		\mitem  \en{}
		\E{\e \e \e }
		%%%%%%%%%%%%%%%%%%%%%%%%%%%%
		\mitem  \en{}
		\E{\e \e \e }
		%%%%%%%%%%%%%%%%%%%%%%%%%%%%	
		\mitem  \en{}
		\E{\e \e \e }
		%%%%%%%%%%%%%%%%%%%%%%%%%%%%
	\end{mainenum}
\end{Lemma}
\begin{usage}{}
	
\end{usage}
%%%%%%%%%%%%%%%%%%%%%%%%%%%%
%%%%%%%%%%%%%%%%%%%%%%%%%%%%

%usage 須在 Lemma 外













%%%%%%%%%%%%%%%%%%%%%%%%%%%%
%%%%%%%%%%%%%%%%%%%%%%%%%%%%
\begin{Lemma} %
	\centry[][一]{典}[\SB{}][]{}{}
	%%%%%%%%%%%%%%%%%%%%%%%%%%%%
	\Def[1]{} 
	\E{\e \e \e }
	%%%%%%%%%%%%%%%%%%%%%%%%%%%%
	\Def[2]{} 
	\E{\e \e \e }
	%%%%%%%%%%%%%%%%%%%%%%%%%%%%
	\Def[3]{} 
	\E{\e \e \e }
	%%%%%%%%%%%%%%%%%%%%%%%%%%%%
	\Def[4]{} 
	\E{\e \e \e }
	%%%%%%%%%%%%%%%%%%%%%%%%%%%%
	\Def[5]{} 
	\E{\e \e \e }
	%%%%%%%%%%%%%%%%%%%%%%%%%%%%
	\Def[6]{} 
	\E{\e \e \e }
	%%%%%%%%%%%%%%%%%%%%%%%%%%%%	
	\Def[7]{} 
	\E{\e \e \e }
	%%%%%%%%%%%%%%%%%%%%%%%%%%%%	
	\Def[8]{} 
	\E{\e \e \e }
	%%%%%%%%%%%%%%%%%%%%%%%%%%%%
	\Def[9]{} 
	\E{\e \e \e }
	%%%%%%%%%%%%%%%%%%%%%%%%%%%%	
	\Def[10]{} 
	\E{\e \e \e }
	\you{} 
	\E{\e \e }
	\yin{} 
	\E{\e \e }
	\deff[1]{} 
	\E{\e \e }
	%%%%%%%%%%%%%%%%%%%%%%%%%%%%	
\end{Lemma}
%%%%%%%%%%%%%%%%%%%%%%%%%%%%
%%%%%%%%%%%%%%%%%%%%%%%%%%%%


{\IBMP ●◐◑◒◓}
%%%%%%%%%%%%%%%%%%%%%%%%%%%%
%%%%%%%%%%%%%%%%%%%%%%%%%%%%
\begin{Lemma} %詞條樣例
	\centry[9][⼎]{典}[2312\SB{1}][木廿土]{kiàn}{kon}
%\centry[部首外筆畫][所屬部首]{主字頭}[四角碼][倉頡碼]{國音}{粵音}
%%%%%%%%%%%%%%%%%%%%%%%%%%%%
	\Def[1]{疑問◐}  %義項1(天干計數為甲)
	\E{\e \emph{論語}\emph{里仁}:「\textsf{參}乎!吾。\buttonc{2}道一以貫之。」\e \emph{莊子}\emph{秋水}:「見笑於大方之家。」} %義項用例
	\yin{之吾道一以貫} %引:引申義
	\E{\e \emph{論語}\emph{里仁}:「參乎!吾\textbf{道}一以貫之。」}
	\you{一以吾道貫之} %又:別出義
	\E{\e \emph{論語}\emph{里仁}:「參乎!吾道一以貫之。」}	
	\deff[1]{之吾道一以貫} %\deff子義項,序號為帶圈中文數字
	\E{\e \emph{論語}\emph{里仁}:「參乎!吾\textbf{道}一以貫之。」}	
%%%%%%%%%%%%%%%%%%%%%%%%%%%%
	\Def[2]{表示疑問 Those districts} 
	\E{\e \emph{論語}\emph{里仁}:「\textsf{參}乎!吾道一以貫之。」\e 論語{\IBMCP\textcircled{\footnotesize 引}\textcircled{\footnotesize 又}}里仁:「\textsf{參}乎!吾道一以貫之。」\e \emph{論語}\emph{里仁}:「參乎!吾道一以貫之。」}
%%%%%%%%%%%%%%%%%%%%%%%%%%%%
	\Def[3]{表示疑問 Those districts} 
	\E{\e \emph{論語}\emph{里仁}:「\textsf{參}乎!吾道一以貫之。」\e 論語里仁:「\textsf{參}乎!吾道一以貫之。」\e \emph{論語}\emph{里仁}:「參乎!吾道一以貫之。」}
%%%%%%%%%%%%%%%%%%%%%%%%%%%%
	\Def[4]{表示疑問 Those districts} 
	\E{\e \emph{論語}\emph{里仁}:「\textsf{參}乎!吾道一以貫之。」\e 論語里仁:「\textsf{參}乎!吾道一以貫之。」\e \emph{論語}\emph{里仁}:「參乎!吾道一以貫之。」}
	\you{一以吾道貫之} 
	\E{\e \emph{論語}\emph{里仁}:「參乎!吾道一以貫之。」吾道}
	\yin{之吾道一以貫} 
	\E{\e \emph{論語}\emph{里仁}:「參乎!吾\textbf{道}一以貫之。」}
	\deff[1]{之吾道一以貫} 
	\E{\e \emph{論語}\emph{里仁}:「參乎!吾\textbf{道}一以貫之。」}
%%%%%%%%%%%%%%%%%%%%%%%%%%%%
	\Def[5]{表示疑問 Those districts} 
	\E{\e \emph{論語}\emph{里仁}:「\textsf{參}乎!吾道一以貫之。」\e 論語里仁:「\textsf{參}乎!吾道一以貫之。」\e \emph{論語}\emph{里仁}:「參乎!吾道一以貫之。」}
%%%%%%%%%%%%%%%%%%%%%%%%%%%%
	\Def[6]{表示疑問 Those districts} 
	\E{\e \emph{論語}\emph{里仁}:「\textsf{參}乎!吾道一以貫之。」\e 論語里仁:「\textsf{參}乎!吾道一以貫之。」\e \emph{論語}\emph{里仁}:「參乎!吾道一以貫之。」}
%%%%%%%%%%%%%%%%%%%%%%%%%%%%	
	\Def[7]{表示疑問 Those districts} 
	\E{\e \emph{論語}\emph{里仁}:「\textsf{參}乎!吾道一以貫之。」\e 論語里仁:「\textsf{參}乎!吾道一以貫之。」\e \emph{論語}\emph{里仁}:「參乎!吾道一以貫之。」}
%%%%%%%%%%%%%%%%%%%%%%%%%%%%	
	\Def[8]{表示疑問 Those districts} 
	\E{\e \emph{論語}\emph{里仁}:「\textsf{參}乎!吾道一以貫之。」\e 論語里仁:「\textsf{參}乎!吾道一以貫之。」\e \emph{論語}\emph{里仁}:「參乎!吾道一以貫之。」}
%%%%%%%%%%%%%%%%%%%%%%%%%%%%
	\Def[9]{表示疑問 Those districts} 
	\E{\e \emph{論語}\emph{里仁}:「\textsf{參}乎!吾道一以貫之。」\e 論語里仁:「\textsf{參}乎!吾道一以貫之。」\e \emph{論語}\emph{里仁}:「參乎!吾道一以貫之。」}
%%%%%%%%%%%%%%%%%%%%%%%%%%%%	
	\Def[10]{表示疑問 Those districts} 
	\E{\e \emph{論語}\emph{里仁}:「\textsf{參}乎!吾道一以貫之。」\e 論語里仁:「\textsf{參}乎!吾道一以貫之。」\e \emph{論語}\emph{里仁}:「參乎!吾道一以貫之。」}
%%%%%%%%%%%%%%%%%%%%%%%%%%%%	
\end{Lemma}
%%%%%%%%%%%%%%%%%%%%%%%%%%%%
%%%%%%%%%%%%%%%%%%%%%%%%%%%%

%%%%%%%%%%%%%%%%%%%%%%%%%%%%
%\begin{Lemma} %詞頭無部首信息之樣式
%\centryl{羅}[2312\textsubscript{1}][木廿中水土]{liàn}{lan}
%或歎表讚歎表讚歎表讚歎表讚歎
%\end{Lemma}
%%%%%%%%%%%%%%%%%%%%%%%%%%%%
\npcircds{0}{\ZSKS\textcircled{\footnotesize 引}\textcircled{\footnotesize 又}}
\buttonc{3}\buttonc{4}\buttonc{5}\buttonc{6}\buttonc{7}\buttonc{8}\buttonc{9}\buttonc{10}
%%%%%%%%%%%%%%linfonta 多音節單詞中音符號
{\linfonta \Huge ŧhat thanks\\
	AAAAAAA
	\begin{itemize}[leftmargin={-3pt}]
		\item[.] \'ā=/{\unifont eɪ}/ \={a} as in r\textit{\'ā}d\u{i}\=o
		\item[.] \'ä=/{\unifont ɑː}/ \"{a} as in p\textit{\'ä}ssp\u{o}rt	
		\item[.] \'ă=/{\unifont æ}/ \u{a} as in \textit{\'ă}mplĭf\=y	
		\item[.] \'â=ê/{\unifont ɛ}/ \^{a} as in v\textit{\'â}rĭous
	\end{itemize}
	OOOOOOO
	\begin{itemize}[leftmargin={-3pt}]
		\item[.] \'ŏ=/{\unifont ɒ}/ \u{o} as in \textit{\'ŏ}cç\u{i}d\`{ĕ}nt 
		\item[.] \'ö=/{\unifont ɔː}/ \"{o} as in c\textit{\'ö}rd\u{i}al
		\item[.] \'ō=/{\unifont əʊ}/ \={o} as in s\textit{\'ō}lō		
		\item[.] \'ô=/{\unifont ʌ}/ \^{o} as in m\textit{\'ô}ther		
	\end{itemize}	
	EEEEEEEEE
	\begin{itemize}[leftmargin={-3pt}]
		\item[.] \'ê=/{\unifont ɛ}/ \u{e} as in ĕff\textit{\'ê}ctĭve
		\item[.] \'ē=/{\unifont iː}/ \={e} as in abbr\textit{\'ē}vĭāte
		\item[.] \'ĕ=ĭ/{\unifont ɪ}/ \u{e} as in \textit{ĕ}ff\'êctĭve %字母e發ĭ音時極少在重音位置
	\end{itemize}		
	IIIIIIIII
	\begin{itemize}[leftmargin={-3pt}]
		\item[.] \'ī=/{\unifont aɪ}/ \={i} as in f\textit{\'ī}nănce 
		\item[.] \'ï=/{\unifont iː}/ \={e} as in nä\textit{\'ï}ve
		\item[.] \'ĭ=/{\unifont ɪ}/ \u{i} as in \textit{\'ĭ}dĭot
	\end{itemize}		
	UUUUUUUUU
	\begin{itemize}[leftmargin={-3pt}]
		\item[.] \'û=/{\unifont ʌ}/ \^{u} as in \textit{\'û}plōad
		\item[.] \'ū=/{\unifont juː}/ \={u} as in \textit{\'ū}nĭon
		\item[.] \'ü=/{\unifont uː}/ \"{u} as in s\textit{\'ü}per
		\item[.] \'ŭ=/{\unifont u}/ \u{u} as in p\textit{\'ŭ}ddĭng
	\end{itemize}	
	YYYYYYYYY
	
	\begin{itemize}[leftmargin={-3pt}]
		\item[.] \'ў=ĭ as in s\textit{\'ў}mpathў 
		\item[.] \=y=ī  as in b{\unifont \textbf{\'ȳ}}p\"ass  %Liberation字體無獨立ȳ形
	\end{itemize}	
}


\lettrine{A}{bandon} jsedjk  ioa  abandon of  you hand in this bauf uoi aholics of the day inthe ane of the day
jsedjk  ioa  abandon of  you hand in this bauf uoi aholics of the day inthe ane of the day
jsedjk  ioa  abandon of  you hand in this bauf uoi aholics of the day inthe ane of the day
jsedjk  ioa  abandon of  you hand in this bauf uoi aholics of the day inthe ane of the day
jsedjk  ioa  abandon of  you hand in this bauf uoi aholics of the day inthe ane of the day
\phantomsection
\addcontentsline{toc}{chapter}{序言}
\chapter*{序言}
\chaptermark{序言}

\indent 

時代變遷,轉關盛世,科技昌明。字書編撰,前承先賢職志,近追數字潮流。是以捨部首而取\textsf{倉頡},擇\textsf{英}辭而釋國語。或問,「數字大潮之中編撰辭書何爲?」曰一爲查檢,一爲閱讀也。

 Clarendon TTFl English steady Dictionary Press advance to authorize has towards made the it completion possible preparation fo r of the and the Delegates issue great of Oxford
 o this f the book which in its own province and on its own scale uses the materials and
follow s the methods b y which the O xford editors have revolution­
ized lexicography. The book is designed as a dictionary, and not
as an encyclopaedia; that is, the uses o f words and phrases as such
are its subject matter, and it is concerned w ith givin g information
about the things fo r which those words and phrases stand only so
fa r as correct use of the words depends upon knowledge of the
things. The degree of this dependence varies greatly with the
kind of w ord treated, the difference between cyclopaedic and dic­
tionary treatment varies w ith it, and the line of distinction is
accordingly a fluctuating and dubious one. It is to the endeavour to
discern and keep to this line that w e attribute whatever peculiarities
w e are conscious of in this dictionary as compared w ith others of the
same size. One o f these peculiarities is the large amount o f space
given to the common words that no one goes through the day w ith­
out using scores or hundreds of times, often disposed of in a line or
two on the ground that they are plain and simple and that every one
knows all about them b y the light of nature, but in fact entangled
w ith other words in so m any alliances and antipathies during their
perpetual knocking about the w orld that the idiomatic use o f them
is fa r from easy; chief am ong such words are the prepositions, the
conjunctions, the pronouns, and such 4 sim ple1 norms and verbs as
hand and way, go and p u t. Another peculiarity is the use, copious
for so small a dictionary, o f illustrative sentences as a necessary
supplement to definition when a w ord has different senses between
which the distinction is fine, or when a definition is obscure and
unconvincing until exem plified; these sentences often are, but still
more often are not, quotations from standard authors; they are
meant to establish the sense of the definition b y appeal not to ex­
ternal authority, but to the reader’s own consciousness, and there­
fore their source, even when authoritative, is not named. A third
and a fourth peculiarity are the direct results of the preceding ones;
if common words are to be treated at length, and their uses to be
copiously illustrated, space must be saved both b y the curtest

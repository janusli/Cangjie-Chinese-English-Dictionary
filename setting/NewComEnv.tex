%%%定義新命令時m對應命令後{}的個數o對應命令後[]的個數
%% Helper macro for lemma signs.
%% This composes the lemma sign and optional homonym number as a single textual unit.
%#2右下垂
\NewDocumentCommand \thelemmasign {m g} {#1\IfValueTF{#2}{\textsubscript{#2}}{}}


%\NewDocumentCommand \thelemmasign {m g g} {#1{\textsubscript{#2}}\textsuperscript{#3}}{}}  %\IfValueTF{#3}{\textsuperscript{#3}}
%% The lemma sign. Only for top level lemmata.
%%
%% Args: LemmaID, LemmaSign, [HomonymNumber]
%%
%% Magic includes making a hyperref target for the lemma and updating the header.

%% The Lemma environment surrounds every entry.該環境可以不用在每個詞條,但爲方便源文件詞條摺疊,故套在每個詞條上。也可備擴展功能用。
	\NewDocumentEnvironment {Lemmah} {}
	{\par%
		\noindent%
		\ignorespaces}%
	{\ignorespacesafterend
		}

%%%%%%%%%%%%%%%%%%%%%%%%%%%%%%%%%%%%%%%
%% The Lemma environment surrounds every entry.
%%% 非colorbox詞條懸掛縮進
%% This is the basic entry indentation length.
\newlength{\basicentryindent}
\setlength{\basicentryindent}{0em}%0.625em %0.23em
%% This length will be mathematically manipulated for increasing indentation
%% when nesting beneath lemmas (VerbTheme, Subentry, etc.).
\newlength{\entryindent}
\setlength{\entryindent}{\basicentryindent}

%% Entry indentation mechanism.
%%
%% This uses the experimental LaTeX3 syntax.
\ExplSyntaxOn
%% The entry depth number. This is used by the dot printing, but not by the actual indentation.
\int_zero_new:N \entrydepth
%% Increase the entry depth number.
\NewDocumentCommand{\increntrydepth} {} {\int_incr:N \entrydepth}
%% Reset the entry depth number. Used in each Lemma environment.
\NewDocumentCommand{\resetentrydepth} {} {\int_zero:N \entrydepth}
%% Print appropriate depth of dots from the margin.
\NewDocumentCommand{\printentrydepthdots} {}
% \fakebox[...]{...} is defined in crippenmacros.sty, it creates a visible but widthless box.
{\fakebox{%
		% Go back to the margin.
		%\hspace*{-\hangindent}\hspace*{\basicentryindent}
		% Initialize a temporary loop variable. \l_tmpa_int is a standard LaTeX3 temp var.
		\int_set_eq:NN {\l_tmpa_int} {\entrydepth}
		% Loop over the entry depth number until it's zero.
		\int_do_while:nNnn {\l_tmpa_int} > {\c_zero}
		% Print a dot and space.
		%% FIXME: This should use \basicentryindent but that’s too wide for some reason.
		{·\hspace*{0em} %0.5em
			% Decrement the loop variable.
			\int_decr:N \l_tmpa_int}}}
\ExplSyntaxOff
%% The Lemma environment surrounds every entry.
\NewDocumentEnvironment {Lemma} {} %詞條懸掛縮進,用於非colorbox詞條
{%\par%
	% Reset the entry depth number.
	\resetentrydepth%
	% The memoir commands change other things so we use plain TeX hanging.
	% This may be a bug in memoir, or a misunderstanding on my part. But this works.
	% Set the hanging indentation length.
	\hangindent=\entryindent%
	% Hang after the first line.
	\hangafter=1	% Suppress normal first-line indenting.
	%\noindent%
	\ignorespaces}%
{\par\ignorespacesafterend}%\par
%%%%%%%%%%%%%%%%%%%%%%%%%%%%%%%%%%%%%%%%%%%%%%%%%

%%%%%%%%%%%%%%%%%%%%%%%%%%%
%%%%%%%%%%%%%%%%%%%%%%%%%%%
\usetikzlibrary{arrows.meta}

%常用字 myred
\newcommand\buttonleftsupera[1]{  %漢字左上角部首外筆畫數目標示
	\begin{tikzpicture}
	\draw [draw=white,line width=0cm,
	-{Fast Triangle[length=0mm,inset=0mm]}] %fill=white,
	(0,0) -- (0,0);  
	\pgftext[base,x=.7cm,y=.4cm]{\fontsizec{.5cm}\color{myred}\ncircds{#1}}  
	\end{tikzpicture}
}

\newcommand\buttonleftsuba[1]{
	\begin{tikzpicture}
	\draw [draw=white,line width=0cm,
	-{Fast Triangle[length=0mm,inset=0mm]}] %fill=white,
	(0,0) -- (0,0);  %
	\draw [draw=myred] (.4,.05) circle (.22cm); 
	\pgftext[base,x=.4cm,y=-0.096cm]{\ZSKS\fontsizec{.37cm}\color{myred} #1} 
	\end{tikzpicture}
}

%次常用字 GeRed
\newcommand\buttonleftsuperb[1]{  %漢字左上角部首外筆畫數目標示
	\begin{tikzpicture}
	\draw [draw=white,line width=0cm,
	-{Fast Triangle[length=0mm,inset=0mm]}] %fill=white,
	(0,0) -- (0,0);  
	\pgftext[base,x=.7cm,y=.4cm]{\fontsizec{.5cm}\color{GeRed}\ncircds{#1}}  
	\end{tikzpicture}
}

\newcommand\buttonleftsubb[1]{
	\begin{tikzpicture}
	\draw [draw=white,line width=0cm,
	-{Fast Triangle[length=0mm,inset=0mm]}] %fill=white,
	(0,0) -- (0,0);  %
	\draw [draw=GeRed] (.4,.05) circle (.22cm); 
	\pgftext[base,x=.4cm,y=-0.096cm]{\ZSKS\fontsizec{.37cm}\color{GeRed} #1} 
	\end{tikzpicture}
}


%餘字下沉兩行設計 GeRed
\newcommand\buttonleftsuperc[1]{  %漢字左上角部首外筆畫數目標示
	\begin{tikzpicture}
	\draw [draw=white,line width=0cm,
	-{Fast Triangle[length=0mm,inset=0mm]}] %fill=white,
	(0,0) -- (0,0);  
	\pgftext[base,x=.37cm,y=.09cm]{\fontsizec{.43cm}\color{GeRed}\ncircds{#1}}  
	\end{tikzpicture}
}

\newcommand\buttonleftsubc[1]{
	\begin{tikzpicture}
	\draw [draw=white,line width=0cm,
	-{Fast Triangle[length=0mm,inset=0mm]}] %fill=white,
	(0,0) -- (0,0);  %
	 \draw [draw=GeRed] (-0.88,-0.2) circle (.19cm); 
	\pgftext[base,x=-0.88cm,y=-0.3cm]{\ZSKS\fontsizec{.33cm}\color{GeRed} #1} 
	\end{tikzpicture}
}
%%%%%%%%%%%%%同時上標superscript與下標subscript
%\usepackage{fixltx2e}
\usepackage{xcolor}
\def\SPSB#1#2{\rlap{\textsuperscript{\textcolor{red}{#1}}}\SB{#2}}
\def\SP#1{\textsuperscript{#1}}
\def\SB#1{\textsubscript{#1}}


 %古籍最常用漢字
\NewDocumentCommand \centrya {o o m o o m m} %主樣式
{\subsectionmark{[{\ubufontc#4}]{\PFTL#3}} \phantomsection\addcontentsline{toc}{subsection}{#4\textbf{#3}}%   
\index[SJM]{{\ubufont\color{red}\Large#5}![{\ubufontc#4}]{\PFTL#3}} %漢字四角碼索引 
\index[GY]{{\ubufont\color{red}\Large #6}!{\ubufontc#4}{\PFTL#3}[{\ubufontc#7}]} %國音索引
\index[YY]{{\ubufont\color{red}\Large#7}!{\ubufontc#4}{\PFTL#3}[{\ubufontc#6}]} %粵音索引
\index[CJ]{{\ubufontc #4}{\PFTL #3}} %倉頡索引
\index[BH]{{\color{red}\fontsizec{0.7cm}\ncircds{#1}}![{\ubufontc#4}]{\PFTL#3}} %總筆畫數索引
\thelemmasign{\hypertarget{#3}{\lettrine[lines=3,lhang=0.39,loversize=-0.2]{\hspace{0.15cm}\SPSB{\buttonleftsupera{#1}}{\buttonleftsuba{#2}}\hspace{-0.33cm}\color{myred}\IBMP #3}{} }} {{\color{myred}{\ubufontc #4},{\ubufontc #5},{\ubufontc #6},{\ubufontc #7}。}}}


%次常用字
\NewDocumentCommand \centryb {o o m o o m m} %主樣式
{\subsectionmark{[{\ubufontc#4}]{\PFTL#3}} \phantomsection\addcontentsline{toc}{subsection}{#4\textbf{#3}}%   
	\index[SJM]{{\ubufont\color{red}\Large#5}![{\ubufontc#4}]{\PFTL#3}} %漢字四角碼索引 
	\index[GY]{{\ubufont\color{red}\Large #6}!{\ubufontc#4}{\PFTL#3}[{\ubufontc#7}]} %國音索引
	\index[YY]{{\ubufont\color{red}\Large#7}!{\ubufontc#4}{\PFTL#3}[{\ubufontc#6}]} %粵音索引
	\index[CJ]{{\ubufontc #4}{\PFTL #3}} %倉頡索引
	\index[BH]{{\color{red}\fontsizec{0.7cm}\ncircds{#1}}![{\ubufontc#4}]{\PFTL#3}} %總筆畫數索引
	\thelemmasign{\hypertarget{#3}{\lettrine[lines=3,lhang=0.39,loversize=-0.2]{\hspace{0.15cm}\SPSB{\buttonleftsuperb{#1}}{\buttonleftsubb{#2}}\hspace{-0.33cm}\color{myred}\IBMP #3}{} }} {{\color{myred}{\ubufontc #4},{\ubufontc #5},{\ubufontc #6},{\ubufontc #7}。}}}


%又次漢字(備檢索而不索引國音與粵音)
\NewDocumentCommand \centry {o o m o o m m} %詞條主樣式
{\subsectionmark{[{\ubufontc#4}]{\PFTL#3}} \phantomsection\addcontentsline{toc}{subsection}{#4\textbf{#3}}%   
\index[SJM]{{\ubufont\color{red}\Large#5}![{\ubufontc#4}]{\PFTL#3}} %漢字四角碼索引 
%\index[GY]{{\ubufont\color{red}\Large #6}!{\ubufontc#4}{\PFTL#3}[{\ubufontc#7}]} %國音索引
%\index[YY]{{\ubufont\color{red}\Large#7}!{\ubufontc#4}{\PFTL#3}[{\ubufontc#6}]} %粵音索引
\index[CJ]{{\ubufontc #4}{\PFTL #3}} %倉頡索引
\index[BH]{{\color{red}\fontsizec{0.7cm}\ncircds{#1}}![{\ubufontc#4}]{\PFTL#3}} %總筆畫數索引
	\thelemmasign{\hypertarget{#3}{\lettrine[lines=2,lhang=0.43,loversize=-0.2]{\hspace{0.15cm}\SPSB{\buttonleftsuperc{#1}}{\buttonleftsubc{#2}}\hspace{-0.87cm}\color{GeRed}\IBMP #3}{} }} {{\color{GeRed}{\ubufontc #4},{\ubufontc #5},{\ubufontc #6},{\ubufontc #7}。}}}
 
 
%%%%%%%%%%音項劃分
\newcommand\pro[1]{\eline{#1}}
%%%%%%%%%%%%%%義項
%\NewDocumentCommand \Def { o m } %%天干計數主義項,10項以內,華康綜藝字體
%{{\linfontn\buttonh{#1}\hspace*{-0.4em}\IBMP #2}}% \IBMP \hspace*{-0.57em}
\NewDocumentCommand \Def { o m } %%天干計數主義項,10項以內,華康綜藝字體
{{\fontsizec{0.32cm}\color{mypurple}\ncircbrs{#1}} #2}% {\fontsizec{0.42cm}\HTC #2} \hspace*{-0.57em}\libcircblk\ipacircnum{#1}\fontsizec{0.417cm}

\NewDocumentCommand \deff {o m}%%中文計數子義項,26項以內,方框大寫字母
{{\fontsizec{0.32cm}\color{mypurple}\Alphcircdrsu{#1} }#2}% 


\newcommand{\you}[1]{{\fontsizec{0.32cm}\color{black}\Alphcircdrsl{1}}#1} %主義項之引申義Also...
\newcommand{\yin}[1]{{\fontsizec{0.32cm}\color{black}\Alphcircdrsl{8}}#1}%主義項之別出義Hence...

%%%%%%%%%%%%%%%%%%%%%%%%%
%%%%%%% Example簡易例證命令
\NewDocumentCommand \E{m}%non-itemized example 首字前不自動忽略空格,源文件需注意。
{\IBMP #1}  %\IBMP
\newcommand\e{\ding{45}} %\e手型符號

%%%%%%%%%%%%%%%%%%%%%%%%%%%
%%%詞條內短語,亦需自動鏈接與自動索引。(條前標籤可複製作它用)


%%詞源說明環境(暫時不用)
%%\d derive from latin, greek etc
%\renewcommand \d {\color{myviolet}\ding{218}}
%\NewDocumentCommand \Etymology {m} {[\ding #1]} %帶箭頭
%\NewDocumentCommand \Etymology {m} {[#1]}
%%inflection說明環境\fif{}
%\renewcommand \c {{\color{myviolet}\ding{49}}}
%\NewDocumentCommand \fif {m} {\c %\color{main}\textbf{\textit{#1}}}

\NewDocumentCommand \Etymology {m} {{\HBMBCP #1}}

%%%%%定義簡易參照命令
\renewcommand{\textsl}[1]{\hyperlink{#1}{\color{black}\thelemmasign{#1}}\index[TH]{{\color{red}\Large\HGB #1}!\subsectiontitle}}

\NewDocumentCommand \Ref{m}%無指頭互參用
{\hyperlink{#1}{\textbf{\thelemmasign{#1}}}\index[TH]{{\color{red}\Large\HGB #1}!\subsectiontitle}}  %標記所在詞頭
\NewDocumentCommand \reff{m}%正文行文thesaurus互參用標記所在複音詞,複音詞均已標記為subsectionmark
{\hyperlink{#1}{\textbf{\thelemmasign{#1}}}\index[TH]{{\color{red}\Large\HGB #1}!\subsectiontitle}}  %標記所在短語

\NewDocumentCommand \Reff{m}%無指頭互參用
{\hyperlink{#1}{\textbf{\thelemmasign{#1}}}\index[TH]{{\color{red}\Large\HGB #1}!\subsectiontitle}}

\newcommand{\en}[1]{\index[EC]{{\color{red}\Large\ubufontc #1}!\subsectiontitle}\ubufontc #1}  %同時生成簡明英漢辭典


%%%%%%%%%%%%%%邊註信息設計
\newcommand{\buttonp}[1]%text=white,
{   \begin{tikzpicture}[baseline=(tempname.base)]
	\node[draw=mypurple, fill=white!105,  rounded corners=0.5pt, inner sep=1pt, minimum width=0.9em, minimum height=0.9em, general shadow={fill=gray!50, shadow xshift=0.8pt, shadow yshift=-1.5pt, shadow scale=1.02}] (tempname) {#1};
	\end{tikzpicture}
}
\newcommand \syn [1] {{\color{myviolet}$\thickapprox$\Reff{#1}}}%同義詞Synonym $\thickapprox$
\newcommand \ant [1] {{\color{myviolet}$\neq$\Reff{#1}}}%反義詞antonym ≠ $\neq$
\newcommand \hyper [1] {{\color{myviolet}$\Uparrow$\Reff{#1}}}%上位詞Hypernyms $\Uparrow$
\newcommand \hypo [1] {{\color{myviolet}$\Downarrow$\Reff{#1}}}%下位詞Hyponyms $\Downarrow$

%%%%%%%%%%%%%字母首頁
\newcommand*{\dictchar}[1]{
	\clearpage
	\twocolumn[
	\centerline{\parbox[c][3cm][c]{3cm}{% % or \parbox[t]...
			\centering
			\fontsize{24}{24}
			\selectfont
			{#1}}}]
}

%make index command, always in the last line of the NCE.tex file
%\makeindex
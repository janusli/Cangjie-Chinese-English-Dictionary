%!TEX TS-program = xelatex
%!TEX encoding = UTF-8 Unicode
%中文字體
\usepackage[AutoFallBack]{xeCJK}
%\usepackage{CJKnumb}
\usepackage{zhnumber}
%\zhnumsetup{style={Financial}} %宏包zhnumber-utf8中Financial樣式已修改為十天干計數(甲、乙、丙、丁、戊、己、庚、辛、壬、癸),若更新,則需重新修改宏包源文件。十以上數目及六十干支不知如何映射。。

%Font size customize
\newcommand{\fontsizec}[1]{\fontsize{#1}{\baselineskip}\selectfont}
\linespread{1.5}%行距
%%%中文字體
%中文相鄰標點間距
\xeCJKsetkern{:}{「}{0.2em} %相鄰標點間距
\xeCJKsetkern{:}{『}{0.2em} %相鄰標點間距

\xeCJKsetkern{」}{、}{0.1em} %相鄰標點間距
\xeCJKsetkern{」}{。}{0.1em} %相鄰標點間距
\xeCJKsetkern{」}{;}{0.1em} %相鄰標點間距
\xeCJKsetkern{」}{,}{0.1em} %相鄰標點間距
\xeCJKsetkern{」}{!}{0.1em} %相鄰標點間距
\xeCJKsetkern{」}{?}{0.1em} %相鄰標點間距

\xeCJKsetkern{』}{、}{0.1em} %相鄰標點間距
\xeCJKsetkern{』}{。}{0.1em} %相鄰標點間距
\xeCJKsetkern{』}{;}{0.1em} %相鄰標點間距
\xeCJKsetkern{』}{,}{0.1em} %相鄰標點間距
\xeCJKsetkern{』}{!}{0.1em} %相鄰標點間距
\xeCJKsetkern{』}{?}{0.1em} %相鄰標點間距

\xeCJKsetkern{。}{」}{0.1em} %相鄰標點間距
\xeCJKsetkern{!}{」}{0.1em} %相鄰標點間距
\xeCJKsetkern{?}{」}{0.1em} %相鄰標點間距
\xeCJKsetkern{』}{」}{0.1em} %相鄰標點間距

\xeCJKsetkern{。}{』}{0.1em} %相鄰標點間距
\xeCJKsetkern{!}{』}{0.1em} %相鄰標點間距
\xeCJKsetkern{?}{』}{0.1em} %相鄰標點間距

\xeCJKsetkern{「}{『}{0.1em} %相鄰標點間距

\xeCJKsetkern{)}{、}{0.1em} %相鄰標點間距
\xeCJKsetkern{)}{。}{0.1em} %相鄰標點間距
\xeCJKsetkern{)}{;}{0.1em} %相鄰標點間距
\xeCJKsetkern{)}{,}{0.1em} %相鄰標點間距
\xeCJKsetkern{)}{!}{0.1em} %相鄰標點間距
\xeCJKsetkern{)}{?}{0.1em} %相鄰標點間距
\xeCJKsetkern{)}{:}{0.1em} %相鄰標點間距

\renewcommand{\CJKglue}{\hskip 0pt plus 0.07\baselineskip} %漢字字距調整
\renewcommand{\CJKecglue}{\hskip 0pt plus 0.17\baselineskip} %漢字與西文及Arabic Number之字間距調整

%%中文字體設置
\setCJKmainfont[BoldFont=H-BuMing-Bold,ItalicFont=DFHKStdKai-B5CP]{I.BMingCP}  % .Heiti GB18030PUA %IPAGothic %I.BMingCP %IPAGothicBlackRoundSquare 為正常使用IPAGothic字體20以上的帶圈數字
%[BoldFont=H-BuMing-Bold,ItalicFont=DFHKStdKai-B5CP]
%\renewcommand{\textbf}[1]{{\linfonta\HBMB #1}}
%\renewcommand{\textit}[1]{{\linfontd\DHKCP #1}}
\setmainfont{Liberation Serif}

\setCJKfallbackfamilyfont{\CJKrmdefault}[BoldFont=]
{ [SlantedFont=]{HanaMinA} ,     %HanaMinA
	[BoldFont=]          {HanaMinB} }   %HanaMinB

\PassOptionsToPackage{CJKchecksingle}{xeCJK}%%CJKchecksingle衝突解決,但似乎未起作用。
\xeCJKsetup{CheckSingle=true,PunctStyle=banjiao,RubberPunctSkip=false}%AutoFakeBold=2,僞粗體選項 %PunctWidth=0.3em, 標點距離固定值選項
\XeTeXlinebreaklocale"zh_TW"
\XeTeXlinebreakskip = 0pt plus 1pt



\usepackage{xeCJKfntef}[2014/11/05]
\usepackage{fix-cm}
\xeCJKsetup{%
	underwave/symbol=
	\fontsize{0.3em}{0pt}%
	\fontencoding{U}\fontfamily{lasy}\selectfont
	\char 58\relax}
\xeCJKsetup { underwave = { format = \color{black}, depth=0.15em} } %depth值為底線與字體間距
\xeCJKsetup { underline = { format = \color{black}, depth=0.2em, thickness=0.3pt} }
\newcommand*\W{\CJKunderwave*-} %可斷下浪線"-"前星號使浪線不跨標點
\newcommand*\Z{\CJKunderline-} %可斷下直線
\renewcommand{\textsf}{\Z}%重新定義\textsf命令使ctrl+shift+a快捷鍵輸入專名直線以提高排版效率
\renewcommand{\emph}{\W}%重新定義\textsf命令使ctrl+shift+e快捷鍵輸入專名浪線以提高排版效率

%%%%%中文常用字型定義
%%%%%%%%%%%%%
%%%%%%%%%中文字體
%%%%%%%%宋體仿宋體

\setCJKfamilyfont{WYFS}{WenyueType GutiFangsong (Noncommercial Use)}% 文悅聚珍仿宋體
\newcommand{\WYFS}{\CJKfamily{WYFS}}% 文悅聚珍仿宋體

\setCJKfamilyfont{ZSKS}{Chekiang Shu Ke Sung}% 浙江書刻宋體
\newcommand{\ZSKS}{\CJKfamily{ZSKS}}% 浙江書刻宋體


%%%%%%%%%%%%%%%
%%%%%%%%%%%%%明體

\setCJKfamilyfont{IBMP}{I.BMingCP}	% I.BMing新明體
\newcommand{\IBMP}{\CJKfamily{IBMP}}    %I.BMing日文新明體

\setCJKfamilyfont{KXD}{TypeLand KhangXi Dict}	% 康熙字典體
\newcommand{\KXD}{\CJKfamily{KXD}}    %康熙字典體

\setCJKfamilyfont{HNMA}{HanaMinA}	% 花園明朝體A http://fonts.jp/hanazono/
\newcommand{\HNMA}{\CJKfamily{HNMA}}    %花園明朝體A
%花園明朝A(HanaMinA.ttf)
%46,515字(非漢字 6,702字、漢字 31,573字、IVD異体字 8,240字。フォント認識用のU+20000を除く)
%非漢字
%%CJK統合漢字(URO、URO+、Ext-A)
%%CJK互換漢字(補助を含む)およびその正規化符号位置
%%JIS X 0213:2004収録の漢字
%%IVD(およびSIPのベースグリフ)
%通用規範漢字表(通用规范汉字表)収録の漢字
%%HKSCS(香港増補字符集)収録の漢字 

\setCJKfamilyfont{HNMB}{HanaMinB}	% 花園明朝體B http://fonts.jp/hanazono/
\newcommand{\HNMB}{\CJKfamily{HNMB}}    %花園明朝體B
%52,844漢字(フォント認識用のU+4E00、フォント名の「花」「園」「明」「朝」を除く)およびASCII
%%CJK統合漢字(Ext.B、Ext.C、Ext.D、Ext.E) 


\setCJKfamilyfont{HBM}{H-BuMing-Regular} % 不明體
\newcommand{\HBM}{\CJKfamily{HBM}} 

\setCJKfamilyfont{HBMB}{H-BuMing-Bold} % 不明體粗
\newcommand{\HBMB}{\CJKfamily{HBMB}} 

\setCJKfamilyfont{HBMBCP}{H-BuMing-BoldCP} % 不明體粗標點修改版
\newcommand{\HBMBCP}{\CJKfamily{HBMBCP}} 

\setCJKfamilyfont{HMOP}{H-MingLiU-UN-03} % 舊細明體
\newcommand{\HMOP}{\CJKfamily{HMOP}} 


%%%%%%%%%%%%%
%%%%%%%%%%%楷體

\setCJKfamilyfont{DHKCP}{DFHKStdKai-B5CP}  % 華康香港標準楷體標點修正版
\newcommand{\DHKCP}{\CJKfamily{DHKCP}}      % 華康香港標準楷體標點修正版

%%%%%%%%%%%%
%%%%%%%%%%黑體

\setCJKfamilyfont{HGB}{HGBLight} %% 冬青國標輕黑繁
\newcommand{\HGB}{\CJKfamily{HGB}}    %%冬青國標輕黑繁
%\setCJKfamilyfont{HGBCP}{HGBLightCP} %% 冬青國標輕黑繁
%\newcommand{\HGBCP}{\CJKfamily{HGBCP}}    %%冬青國標輕黑繁

\setCJKfamilyfont{HGBT}{HGBThin} %% 冬青國標細黑繁
\newcommand{\HGBT}{\CJKfamily{HGBT}}    %%冬青國標細黑繁

\setCJKfamilyfont{HGBM}{HGBMedium} %% 冬青國標中黑繁
\newcommand{\HGBM}{\CJKfamily{HGBM}}    %%冬青國標中黑繁

\setCJKfamilyfont{FZCCH}{FZChaoCuHei-M10} %% 方正超粗黑繁
\newcommand{\FZCCH}{\CJKfamily{FZCCH}}    %%方正超粗黑繁

\setCJKfamilyfont{FZCH}{FZCuHei-B03} %% 方正粗黑繁
\newcommand{\FZCH}{\CJKfamily{FZCH}}    %%方正粗黑繁

\setCJKfamilyfont{PFT}{PingFang TC Regular} %%平方黑臺灣標準常規
\newcommand{\PFT}{\CJKfamily{PFT}}    %%平方黑臺灣標準常規

\setCJKfamilyfont{PFTL}{PingFang TC Light} %%平方黑臺灣標準輕
\newcommand{\PFTL}{\CJKfamily{PFTL}}    %%平方黑臺灣標準常輕

%%%%%%%%%%%%%%
%%%%%%%%書家系列字體
%\setCJKfamilyfont{TYZK}{A-OTF Outai Kaisho Std} %%田英章楷書大字集名稱修正版
%\newcommand{\TYZK}{\CJKfamily{TYZK}}    %%田英章楷書大字集需繁體輸入
\setCJKfamilyfont{TYZK}{TYZKai} %%田英章楷書大字集名稱修正版
\newcommand{\TYZK}{\CJKfamily{TYZK}}    %%田英章楷書大字集需繁體輸入

\setCJKfamilyfont{HNL}{HannotateTCL} %%華康手札體繁Regular名稱修改版
\newcommand{\HNL}{\CJKfamily{HNL}}   %%華康手札體Regular繁名稱修改版

\setCJKfamilyfont{HNB}{HannotateTCB} %%華康手札體繁Bold名稱修改版
\newcommand{\HNB}{\CJKfamily{HNB}}   %%華康手札體Bold繁名稱修改版

\setCJKfamilyfont{HZL}{HanzipenTCL}    %%華康翩翩體繁Regular名稱修改版
\newcommand{\HZL}{\CJKfamily{HZL}}   %%華康翩翩體繁Regular名稱修改版

\setCJKfamilyfont{HZB}{HanzipenTCB}    %%華康翩翩體繁Bold名稱修改版
\newcommand{\HZB}{\CJKfamily{HZB}}   %%華康翩翩體繁Bold名稱修改版

%%%%%%%%%%%%%%%
%%%%%%%%%英文常用字体

%Linux Liberation Fonts
\newfontfamily\linfont{Liberation Serif}
\newfontfamily\linfonti{LiberationSerif-Italic}
\newfontfamily\linfontb{LiberationSerif-Bold}
\newfontfamily\linfontbi{LiberationSerif-BoldItalic}
\newfontfamily\linfontsb{LiberationSans-Bold}
\newfontfamily\linfontsnb{LiberationSansNarrow-Bold}


%Linux Libertine Fonts
\newfontfamily\lixfont{Linux Libertine}
\newfontfamily\lixfonti{Linux Libertine Italic}
\newfontfamily\lixfontb{Linux Libertine Bold}
\newfontfamily\lixfontbi{Linux Libertine Bold Italic}
\newfontfamily\lixfontsb{Linux Libertine Semibold}
\newfontfamily\lixfontsbi{Linux Libertine Semibold Italic}
\newfontfamily\lixfontd{Linux Libertine Display}

%Linux BioLinum Fonts
\newfontfamily\libfont{Linux Biolinum}
\newfontfamily\libfonti{Linux Biolinum Italic}
\newfontfamily\libfontb{Linux Biolinum Bold}
\newfontfamily\libfontk{Linux Biolinum Keyboard}

%ubuntu font family
\newfontfamily\ubufont{Ubuntu} %ubuntu regular
\newfontfamily\ubufonti{Ubuntu Italic}
\newfontfamily\ubufontb{Ubuntu Bold}
\newfontfamily\ubufontm{Ubuntu Medium}
\newfontfamily\ubufontl{Ubuntu Light}
\newfontfamily\ubufontc{Ubuntu Condensed}

%Proforma[1988 - Petr van Blokland] SC Font Series
\newfontfamily\profonta{Proforma-BoldSC}
\newfontfamily\profontb{Proforma-BoldItalicSC}
\newfontfamily\profontc{Proforma-MediumSC}
\newfontfamily\profontd{Proforma-MediumItalicSC}
\newfontfamily\profonte{Proforma-BookSC}
\newfontfamily\profontf{Proforma-BookItalicSC}
\newfontfamily\tpfont{Trajan Pro 3}
\newfontfamily\tpfontb{Trajan Pro 3 Bold}


%Junicode fonts
\newfontfamily\unifont{Junicode}
\newfontfamily\unifontb{Junicode-Bold}
\newfontfamily\unifonti{Junicode-Italic}
\newfontfamily\unifontbi{Junicode-BoldItalic}

%Century Schoolbook
\newfontfamily\csbfont{Century Schoolbook}
\newfontfamily\csbfontb{Century Schoolbook Bold}
\newfontfamily\csbfonti{Century Schoolbook Italic}
\newfontfamily\csbfontbi{Century Schoolbook Bold Italic}

%%Cambria 
\newfontfamily\camfont{Cambria}
\newfontfamily\camfontb{Cambria Bold}
\newfontfamily\camfonti{Cambria Italic}
\newfontfamily\camfontbi{Cambria Bold Italic}
\newfontfamily\camfontm{Cambria Math}

%%Minion Pro 
\newfontfamily\mpfont{Minion Pro}
\newfontfamily\mpfontb{Minion Pro Bold}
\newfontfamily\mpfonti{Minion Pro Italic}
\newfontfamily\mpfontbi{Minion Pro Bold Italic}
%
\newfontfamily\mpfontbc{MinionPro-BoldCn}
\newfontfamily\mpfontbci{MinionPro-BoldCnIt}
\newfontfamily\mpfontm{MinionPro-Medium}
\newfontfamily\mpfontmi{MinionPro-MediumIt}
\newfontfamily\mpfontsb{MinionPro-Semibold}
\newfontfamily\mpfontsbi{MinionPro-SemiboldIt}

%%PT Russian Fonts
\newfontfamily\ptfont{PT Serif}
\newfontfamily\ptfontb{PT Serif Bold}
\newfontfamily\ptfonti{PT Serif Italic}
\newfontfamily\ptfontbi{PT Serif Bold Italic}
\newfontfamily\ptfontc{PT Serif Caption}
\newfontfamily\ptfontci{PT Serif Caption Italic}
\newfontfamily\ptfonts{PT Sans}
\newfontfamily\ptfontsb{PT Sans Bold}
\newfontfamily\ptfontsi{PT Sans Italic}
\newfontfamily\ptfontsc{PT Sans Caption}
\newfontfamily\ptfontscb{PT Sans Caption Bold}
\newfontfamily\ptfontsn{PT Sans Narrow}
\newfontfamily\ptfontsnb{PT Sans Narrow Bold}
\newfontfamily\ptfontm{PT Mono}
\newfontfamily\ptfontmb{PT Mono Bold}

%Akzidenz Grotesk[1966 - Günter Gerhard Lange]
\newfontfamily\akzfont{Akzidenz-Grotesk BQ Extra Bold Condensed}

%Luminari
\newfontfamily\lumfont{Luminari}

%Lithos Pro Regular 
\newfontfamily\litfont{LithosPro-Regular}
\newfontfamily\litfontb{LithosPro-Black}


%Antique Olive[1962 - Roger Excoffon]
\newfontfamily\antfont{AntiqueOliveStd-BoldCond}

%Agenda[1993 - Greg Thompson]
\newfontfamily\agefonta{Agenda-BoldUltraCondensed}
\newfontfamily\agefontb{Agenda-BoldExtraCondensed}
\newfontfamily\agefontc{Agenda-LightUltraCondensed}
\newfontfamily\agefontd{Agenda-LightExtraCondensed}

%%%編號字體
\newfontfamily\numbrs{MnumBRS}
\newfontfamily\numbs{MnumBS}
\newfontfamily\numdrs{MnumDRS}
\newfontfamily\numds{MnumDS}
\newfontfamily\numpy{MnumPYY}
%字母編號字體
\newfontfamily\alphbcu{AlphNumBC} %%圓黑Upper Case
\newfontfamily\alphbcl{AlphNumBCc} %%圓黑lower case
\newfontfamily\alphbru{AlphNumBRS} %%圓角方黑Upper Case
\newfontfamily\alphbrl{AlphNumBRSs} %%圓角方黑lower case
\newfontfamily\alphbsu{AlphNumBS} %%方角方黑Upper Case
\newfontfamily\alphbsl{AlphNumBSs} %%方角方黑lower case
\newfontfamily\alphbsu{AlphNumDRS} %%圓角白底Upper Case
\newfontfamily\alphbsl{AlphNumDRSs} %%圓角白底lower case
\newfontfamily\alphbwcu{AlphNumWC} %%圓形白底Upper Case
\newfontfamily\alphwcl{AlphNumWCc} %%圓形白底lower case


%%%%編號字體命令Setting
\newcommand{\ncircbrs}[1]{{\fontspec[Ligatures=Discretionary]{MnumBRS}(#1)}}%%0-100方黑圓角101-200黑圈數字用字體<唯數字93與98都映射為88待解決>
\newcommand{\ncircbs}[1]{{\fontspec[Ligatures=Discretionary]{MnumBS}(#1)}}%%0-100黑方框101-200黑圈數字用字體<唯數字93與98都映射為88待解決>
\newcommand{\ncircdrs}[1]{{\fontspec[Ligatures=Discretionary]{MnumDRS}(#1)}}%%0-100圓角方框101-200黑圈數字用字體<唯數字93與98都映射為88待解決>
\newcommand{\ncircds}[1]{{\fontspec[Ligatures=Discretionary]{MnumDS}(#1)}}%%0-100圓角方方框101-200黑圈數字用字體<唯數字93與98都映射為88待解決>

%%
%%%%%
\newcommand{\Alphcircbcu}[1]{{\fontspec[Ligatures=Discretionary]{AlphNumBC}(#1)}}%%%%1-26圓黑大寫Upper字母,其餘數字同Mnum系列
\newcommand{\Alphcircbcl}[1]{{\fontspec[Ligatures=Discretionary]{AlphNumBCc}(#1)}}%%%%1-26圓黑小寫lower字母,其餘數字同Mnum系列
\newcommand{\Alphcircbrsu}[1]{{\fontspec[Ligatures=Discretionary]{AlphNumBRS}(#1)}}%%1-26圓角方形黑底大寫Upper字母,其餘數字同Mnum系列
\newcommand{\Alphcircbrsl}[1]{{\fontspec[Ligatures=Discretionary]{AlphNumBRSs}(#1)}}%%%%1-26圓角方形黑底小寫lower字母,其餘數字同Mnum系列
\newcommand{\Alphcircbsu}[1]{{\fontspec[Ligatures=Discretionary]{AlphNumBS}(#1)}}%%1-26圓角方形黑底大寫Upper字母,其餘數字同Mnum系列
\newcommand{\Alphcircbsl}[1]{{\fontspec[Ligatures=Discretionary]{AlphNumBSs}(#1)}}%%%%1-26圓角方形黑底小寫lower字母,其餘數字同Mnum系列
\newcommand{\Alphcircdrsu}[1]{{\fontspec[Ligatures=Discretionary]{AlphNumDRS}(#1)}}%%1-26圓角方形白底大寫Upper字母,其餘數字同Mnum系列
\newcommand{\Alphcircdrsl}[1]{{\fontspec[Ligatures=Discretionary]{AlphNumDRSs}(#1)}}%%%%1-26圓角方形白底小寫lower字母,其餘數字同Mnum系列
\newcommand{\Alphcircwcu}[1]{{\fontspec[Ligatures=Discretionary]{AlphNumWC}(#1)}}%%1-26圓形白底大寫Upper字母,其餘數字同Mnum系列
\newcommand{\Alphcircwcl}[1]{{\fontspec[Ligatures=Discretionary]{AlphNumWCc}(#1)}}%%%%1-26圓形白底小寫lower字母,其餘數字同Mnum系列
%!TEX TS-program = xelatex
%!TEX encoding = UTF-8 Unicode
%% XeLaTeX font support.
\usepackage{fontspec,xltxtra,lipsum}
\renewcommand{\lipsum}{【因字頭設計爲下沉三行,這裏需填寫字頭補充信息,如「異體字」、「六書」、「韻書」、「說文」之類。】} %實際排版時\lipsum命令應查找替換爲空。
%%%% Macrology.
%\usepackage{crippenmacros}
%\setlength{\parindent}{0pt}
%% XeTeX magic. Allows line breaking after en- and em-dashes.
\XeTeXdashbreakstate 1

%%%%
%%%% Packages.
%%%%
\usepackage{amsmath,amssymb}
%% Etoolbox is full of LaTeX programming goodies.
\usepackage{etoolbox}

\usepackage{graphicx} % Required for including pictures
\usepackage{eso-pic}%background pic
\graphicspath{{figures/}} % Specifies the directory where pictures are stored

\usepackage{lettrine} 
%% Mathematical calculations.
\usepackage{calc}

%\usepackage{titlesec,titleps,titletoc}
%\def\contentsname{目~~錄}%
%%%%%%%%%%%%%%%%
%\titleformat{name=\chapter,numberless}[runin] 
%{\normalfont}
%{}{0pt}{}
%\titlespacing{\chapter}
%{0em}{0em}{\wordsep}  %\wordsep
%\settitlemarks{part,chapter,section,subsection,subsubsection}

%\titleformat{\section}[runin] 
%{\normalfont}
%{}{0pt}{}
%\titlespacing{\section}
%{0em}{0em}{\wordsep}  %\wordsep
%\settitlemarks{part,chapter,section,subsection,subsubsection}

%\titleformat{name=\section,numberless}[runin] 
%{\normalfont}
%{}{0pt}{}
%\titlespacing{\section}
%{0em}{0em}{\wordsep}  %\wordsep

%\settitlemarks{part,chapter,section,subsection,subsubsection}

%%%phrase信息
%\titleformat{\subsection}[runin] 
%{\normalfont}
%{}{0pt}{}
%\titlespacing{\subsection}
%{0em}{0em}{\wordsep}  %\wordsep
%\settitlemarks{part,chapter,section,subsection,subsubsection}

%\titleformat{name=\subsection,numberless}[runin] 
%{\normalfont}
%{}{0pt}{}
%\titlespacing{\subsection}
%{0em}{0em}{\wordsep}  %\wordsep

%\settitlemarks{part,chapter,section,subsection,subsubsection}
%%%%%runinhead
%\makeatletter
%\newcommand\runinhead{\@startsection{section*}{4}{\z@}%paragraph
%	{-6\p@}% \p@lus -4\p@ \@minus -4\p@}%
%	{-6\p@}%
%	{\normalfont\normalsize\bfseries\boldmath
%		\rightskip=\z@ \@plus 8em\pretolerance=10000 }}
%\makeatother

%%%%%%%%%%%%%%%
%%Special Signs
\usepackage{bbding,pifont}
\usepackage{wasysym}
%% Colour.
%%% color redefinition
\usepackage{xcolor,tcolorbox}
\definecolor{main}{RGB}{57,77,76}%丹青
\definecolor{myred}{RGB}{255,20,147} %深粉紅
%\definecolor{myred}{RGB}{255,0,255} %洋紅%紫紅
\definecolor{myviolet}{RGB}{199,21,133} %中紫羅蘭紅
\definecolor{mypurple}{RGB}{75,0,130} % 紫色
%\definecolor{mypurple}{RGB}{85,26,139} % 紫色
\definecolor{myblue}{RGB}{0,0,128} % 海軍藍 

\definecolor{DianCyan}{RGB}{8,46,84} % 靛青
\definecolor{DarkBlue}{RGB}{25,25,112} % 深藍
\definecolor{GeRed}{RGB}{227,23,13} % 镉红
\definecolor{IndiaRed}{RGB}{176,23,31} % 印度红
\definecolor{Strawberry}{RGB}{135,38,87} % 草莓色
\definecolor{SoilGreen}{RGB}{56,94,15} % 綠土色
\definecolor{LightViolet}{RGB}{218,112,24} % 淡紫色
\definecolor{SkyBlue}{RGB}{245,255,255} % 天藍色

\pagecolor{SkyBlue} %背景色
%% URL formatting.
\usepackage{url}
\urlstyle{sf}

%% Date and time stuff.
\usepackage{datetime}
\settimeformat{hhmmsstime}

%% Tweak the date and time formats.
\usepackage[UKenglish,cleanlook]{isodate}

%% Various symbols in text mode.
\usepackage{textcomp}

%% Examples.
\let\gla\relax	% conflict with unicode-math
\usepackage{expex}
\lingset{everygla={\normalfont},
	everyglb={\normalfont},
	everyglc={\condensed\small},
	glspace=1ex,
	aboveglbskip=0pt,
	aboveglcskip=-0.0625ex,
	aboveglftskip=0.25ex plus 0.125ex minus 0.125ex,
	belowglpreambleskip=0pt,
	aboveexskip=0.5em plus 0.375em minus 0.25em,
	belowexskip=0.5em plus 0.375em minus 0.25em,
	interpartskip=0.5em plus 0.375em minus 0.25em,
	labeloffset=0.5em,
	textoffset=0.75em,
	mincitesep=0.5em,
	everytrailingcitation={\footnotesize}}
\gathertags
%% FIXME: This is a hack. Somewhere in expex (\judge ?) there's a hardcoded \rm that memoir dislikes.
\renewcommand{\rm}{\rmfamily}
%% FIXME: This is also a hack. Memoir also dislikes a hardcoded \tt which is caused by \getref.
\renewcommand{\tt}{\ttfamily}

%% Magic space handling for commands.
%%
%% Usage: \newcommand{\commandname}{...\xspace}
\usepackage{xspace}

%% Multiple column spanning cells in tables.
%%
%% Usage: \multicolumn{2}{l}{Two columns, left aligned text}
\usepackage{multicol}

%% Multiple row spanning cells in tables.
%%
%% Usage: \multirow{2}{*}{Two rows}
\usepackage{multirow}

%% Big delimiters for multirow in tables.
%%
%% Usage: \ldelim<delimiter>{rows}{width/*}[centred text]
\usepackage{bigdelim}

%% Fancy list environments.
\usepackage{enumitem}
%% No leading between items.
\setenumerate{noitemsep}
\setitemize{noitemsep}
%% Make itemize deeper.
%\renewlist{itemize}{itemize}{7}
%% Change the labels of items.
%\setlist[itemize,1]{label=•}
%\setlist[itemize,2]{label=–}
%\setlist[itemize,3]{label=⁃}
%\setlist[itemize,4]{label=›}
%\setlist[itemize,5]{label=‣}
%\setlist[itemize,6]{label=⁍}
%\setlist[itemize,7]{label=-}

%% Strikeouts and underlines.
%%
%% Note that ‘strikeout’ is \st and not \so.
\usepackage{soul}


%% TeX logos.
\usepackage{metalogo}
\setLaTeXa{\scshape a}
\setlogodrop{0.4375ex}
\setlogokern{Xe}{-0.125ex}
\setlogokern{eL}{-0.1875ex}
\setlogokern{La}{-0.59375ex}
\setlogokern{aT}{-0.1875ex}
\setlogokern{Te}{-0.1875ex}
\setlogokern{eX}{-0.125ex}
\setlogokern{eT}{-0.1875ex}

%%%%%%%%%%%%%%%%%%%%%%%%%%%%%%%
%%%%%%%%%%%%%list environment resetting

%%%%%%%%%%%%%%%%%%%%beamer item style
\usepackage{tikz,tkz-tab}% Cajas de Teoremas, ejemplos, etc.
\usetikzlibrary{positioning,shadows,backgrounds,calc}%
\usepackage{tikzpagenodes}


%%%%%%%
\usepackage{ulem}
%自定義下劃線樣式
\newcommand\eline{\bgroup\markoverwith
	{\textcolor{gray!37}{\rule[-0.4ex]{0.7pt}{1.12em}}}\ULon} %entryline 距離,線長突出,線粗
\newcommand\elinei{\bgroup\markoverwith
	{\textcolor{GeRed}{\rule[-0.41ex]{2.2pt}{14.3pt}}}\ULon} %entryline 距離,線長突出,線粗
\newcommand\elineii{\bgroup\markoverwith
	{\textcolor{myred}{\rule[-0.21ex]{1pt}{40pt}}}\ULon} %entryline 距離,線長突出,線粗
% Listas -- con puntos
\usepackage{enumitem}
\newcommand{\witem}[1]{\item[{\bf #1)}]}
\newcommand{\mitem}[1]{\item \HBMBCP #1} %mainitem
\newcommand{\sitem}[1]{\item \HGB #1} %subitem
\newcommand{\eitem}[1]{\item #1} %exampleitem


%\usepackage{enumerate}





%%%%%%%%%%%%%%%%%%%%%%%%%%%%%%
\usepackage{engord} %1st 3rd之類序數自動生成
%\usepackage{paralist}
\usepackage{enumitem}
%\setenumerate[1]{itemsep=0pt,partopsep=0pt,parsep=\parskip,topsep=5pt}
%\setitemize[1]{itemsep=0pt,partopsep=0pt,parsep=\parskip,topsep=0pt}%item間距設置
%\setdescription{itemsep=0pt,partopsep=0pt,parsep=\parskip,topsep=5pt}
%\renewcommand{\labelitemi}{\textcolor{red}{\textbf{\twonotes}}}

%\usepackage{etoolbox}

\newenvironment{mainenum}[1]{\begin{enumerate}[label=\protect\fontsizec{0.42cm}\color{black}\ncircbs{\arabic*},ref=\arabic*,itemindent=1.1em,noitemsep,labelsep=0.1em,leftmargin=0em,topsep=0pt,partopsep=0pt] %此enum與原\Deff命令中enum一致item序列黑方形
		  #1}
	{\end{enumerate}} %義項細分用環境

%%%%%%%%%%%%%%%%%%%%
%%%tcolorbox and picture in paragraph setting
\usepackage{picinpar}
\usepackage{tcolorbox}
\usepackage{tikz}
% preamble
\tcbuselibrary{breakable}
\tcbuselibrary{skins} 
%%%%%%%%%%%%%%%%%%%%%%%%%%%%





%quote box design
\usepackage{varwidth}
\newtcolorbox{quotebox}[2][]{beamer,skin=enhancedlast jigsaw,interior hidden,
	boxsep=0pt,top=0pt,colframe=red,coltitle=red!50!black,
	fonttitle=\bfseries\sffamily,
	attach boxed title to bottom center,
	boxed title style={empty,boxrule=0.5mm},
	varwidth boxed title=0.5\linewidth,
	underlay boxed title={
		\draw[white,line width=0.5mm]
		([xshift=0.3mm-\tcboxedtitleheight*2,yshift=0.3mm]title.north west)
		--([xshift=-0.3mm+\tcboxedtitleheight*2,yshift=0.3mm]title.north east);
		\path[draw=red,top color=white,bottom color=red!50!white,line width=0.5mm]
		([xshift=0.25mm-\tcboxedtitleheight*2,yshift=0.25mm]title.north west)
		cos +(\tcboxedtitleheight,-\tcboxedtitleheight/2)
		sin +(\tcboxedtitleheight,-\tcboxedtitleheight/2)
		-- ([xshift=0.25mm,yshift=0.25mm]title.south west)
		-- ([yshift=0.25mm]title.south east)
		cos +(\tcboxedtitleheight,\tcboxedtitleheight/2)
		sin +(\tcboxedtitleheight,\tcboxedtitleheight/2); },
	title={#2},#1}

%蛙眼box
\tcbset{frogbox/.style={enhanced,colback=green!10,colframe=green!65!black,
		enlarge top by=5.5mm,
		overlay={\foreach \x in {2cm,3.5cm} {
				\begin{scope}[shift={([xshift=\x]frame.north west)}]
					\path[draw=green!65!black,fill=green!10,line width=1mm] (0,0) arc (0:180:5mm);
					\path[fill=black] (-0.2,0) arc (0:180:1mm);
				\end{scope}}}]}}

\tcbset{beamer,enhanced jigsaw,breakable,colback=red!5!white,colframe=red!75!black,
	fonttitle=\bfseries}


%grammar box design
\usepackage{varwidth}
\newtcolorbox{grambox}[2][]{beamer,enhanced,skin=enhancedlast jigsaw,
	attach boxed title to top left={xshift=-4mm,yshift=-0.5mm},
	fonttitle=\tpfontb,varwidth boxed title=0.7\linewidth,
	colbacktitle=blue!45!white,colframe=red!50!black,
	interior style={top color=blue!10!white,bottom color=red!10!white},
	boxed title style={empty,arc=0pt,outer arc=0pt,boxrule=0pt},
	underlay boxed title={
		\fill[blue!45!white] (title.north west) -- (title.north east)
		-- +(\tcboxedtitleheight-1mm,-\tcboxedtitleheight+1mm)
		-- ([xshift=4mm,yshift=0.5mm]frame.north east) -- +(0mm,-1mm)
		-- (title.south west) -- cycle;
		\fill[blue!45!white!50!black] ([yshift=-0.5mm]frame.north west)
		-- +(-0.4,0) -- +(0,-0.3) -- cycle;
		\fill[blue!45!white!50!black] ([yshift=-0.5mm]frame.north east)
		-- +(0,-0.3) -- +(0.4,0) -- cycle; },
	title={#2},#1}

\newenvironment{usage}[2]  %define usage box USAGE NOTE
{\begin{grambox}[left=1mm,right=1mm,top=1.7mm,bottom=1mm]{\textbf{字辯}:\textbf{#1}}
		\setlength\parindent{1em} #2}
	{\end{grambox}}

\newenvironment{thesaurus}[1] %define Thesaurus box
{\begin{grambox}[left=1mm,right=1mm,top=1.7mm,bottom=1mm]{Thesaurus}
		\setlength\parindent{1em}  #1}
	{\end{grambox}}

\newenvironment{collocations}[1] %define Thesaurus box
{\begin{grambox}[left=1mm,right=1mm,top=1.7mm,bottom=1mm]{Collocations}
		\setlength\parindent{1em} #1}
	{\end{grambox}}

\newenvironment{synonym}[1] %define Synonyms box
{\begin{grambox}[left=1mm,right=1mm,top=1.7mm,bottom=1mm]{Synonyms}
		\setlength\parindent{1em}  #1}
	{\end{grambox}}

\newenvironment{whistory}[1] %define Synonyms box
{\begin{grambox}[left=1mm,right=1mm,top=1.7mm,bottom=1mm]{{\fontsizec{0.6cm} 攷~~證}}
		\setlength\parindent{2em}#1}
	{\end{grambox}}  %%要在Lemma環境之外書寫

%%%%%%%%%%%%
%%subscript and superscript對齊
\usepackage{leftidx}


%%%%%%%%%%%%%%%%%
%插圖及圖文繞排設計
\usepackage{picinpar}


%make index
\usepackage[noautomatic]{imakeidx}%[noautomatic]選項可使正文前printindex(終極定稿),可以針對不同索引使用不同驅動如zhmakeindex, makeindex, xindy之類
%\makeindex[name=ZB,title=字辯索引,intoc,columns=3,columnseprule=true,columnsep=5pt]
\makeindex[name=EC,title=英漢索引,intoc,columns=5,columnseprule=true,columnsep=5pt,options={-s StyleInd.ist}]
\makeindex[name=TH,title=類詞索引,intoc,columns=5,columnseprule=true,columnsep=5pt]%,
\makeindex[name=SJM,title=角碼索引,intoc,columns=5,columnseprule=true,columnsep=5pt,options=-s StyleInd]
\makeindex[name=GY,title=國音索引,intoc,columns=4,columnseprule=true,columnsep=5pt,options=-s StyleInd]
\makeindex[name=YY,title=粵音索引,intoc,columns=4,columnseprule=true,columnsep=5pt,options=-s StyleInd]
\makeindex[name=BH,title=總畫索引,intoc,columns=5,columnseprule=true,columnsep=5pt,options=-s StyleInd]
\makeindex[name=CJ,title=倉頡索引,intoc,columns=6,columnseprule=true,columnsep=5pt,options=-s StyleInd]
%\setcounter{secnumdepth}{3}
%\setcounter{tocdepth}{3}

\makeatletter
\def\@idxitem{\par\addvspace{7\p@ \@plus 3\p@ \@minus 3\p@}\hangindent 17\p@}
\def\subitem{\par\hangindent 10\p@ \hspace*{10\p@}}
\def\subsubitem{\par\hangindent 20\p@ \hspace*{20\p@}}
\def\indexspace{}
\patchcmd\theindex{\indexname}{\indexname\vspace{5pt}}{}{}
\makeatother

%%%%%%%%%%%%%%%%%%
%%splitidx多索引製作方式待研究
%\usepackage{splitidx}
%\newindex[Thesaurus]{Thesaurus}
%\newindex[Phrasal Verbs]{PV}
%\newindex[Idioms]{IDM}
%\newindex[漢英索引]{CE}
%\newindex[Word of Stars]{STARS}

%% Bookmarks.
%%
%% This gives more control over PDF bookmarks, which are used to link to all the sectional divisions.
\usepackage[final]{bookmark}

%% Hyperlinks.
%%
%% Hyperref controls the colours of links (citations, URLs, etc.) which are produced by BibLaTeX.
%% To turn off colouring completely, set both colorlinks=false and pdfborder=[rgb]{0,0,0}.
\usepackage{hyperref}  %若用imakeidx則hyperref需在其後導入
%% Hyperlinks configuration.
\hypersetup{	% Break links across lines into two separate identical links.
	breaklinks=true,
	% Whether to colour link text.
	%colorlinks=false,
	colorlinks=true,
	% Colours of link text {R G B}.
	linkcolor={rgb:blue,1;black,1},
	%linkcolor=black,
	anchorcolor=black,
	%citecolor=green,
	citecolor=black,
	filecolor=cyan,
	menucolor=red,
	pagebackref=true
	runcolor=cyan,
	urlcolor=blue,
	% How links behave when clicked: /O outline, /I inverse, /N nothing, /P "pressed" inset
	% Note that many PDF viewers seem to ignore this.
	pdfhighlight=/P,
	% Thickness of border around links. (An alternative to coloured text.)
	% This is automatically disabled if colorlinks=true.
	%pdfborder={0 0 1},
	pdfborder={0 0 0},
	% Colours of borders, {R G B}
	citebordercolor={0 1 0},
	filebordercolor={1 .5 .5},
	linkbordercolor={1 0 0},
	menubordercolor={1 0 0},
	urlbordercolor={0 0 1},
	runbordercolor={0 .7 .7},
	%% Bookmarks configuration.
	% List section numbers in bookmarks.
	bookmarksnumbered=true,
	% Whether bookmarks are open by default.
	bookmarksopen=true,
	% How deep bookmarks should be open by default.
	bookmarksopenlevel=3,
	pdfencoding=unicode}
